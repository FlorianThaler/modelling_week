\documentclass[aspectratio=43, 11pt]{beamer}

\usetheme{_tugraz_flo}

\usepackage[utf8]{inputenc}
\usepackage[english]{babel}

%% Enter presentation metadata
\title[Short Title]{Computer vision}
\subtitle{Panorama image stitching} 
\author{Florian Thaler}
\date{8. Februar 2025}
\institute{Institute of Visual Computing}
%\instituteurl{www.yourinstitute.tugraz.at}

\begin{document}

	\begin{frame}[plain]
  	\maketitle
	\end{frame}


	%\begin{frame}{Outline}
  %	\tableofcontents
	%\end{frame}

	\section{Einleitung}
	
		\begin{frame}{Motivierendes Beispiel}
	  	...
		\end{frame}


	\section{Computer vision I}

		\begin{frame}
			\frametitle{Computer vision}
			\begin{columns}
				\column{.1\textwidth}
				\column{.4\textwidth}
					\begin{itemize}
		 		  	\item Verarbeitung und Analyse von Bildern
						\item Aufgaben
							\begin{itemize}
								\item Image denoising und image deblurring, ...
								\item Objekterkennung, Bildklassifikation, Segmentierung, ...
								\item ...
							\end{itemize}
					\end{itemize}
				\column{.4\textwidth}
				\column{.1\textwidth}
			\end{columns}
	  	\begin{itemize}
				\item Anwendungsbereiche
					\begin{itemize}
						\item Medizin
						\item Automotive
						\item Logistik
						\item ...
					\end{itemize}
				\item Grundlagendisziplinen von computer vision
					\begin{itemize}
						\item Mathematik und Statistik
						\item Computer science
						\item ...
					\end{itemize}
		  \end{itemize}
		\end{frame}

		\begin{frame}
			\frametitle{Computer vision II}
			\begin{columns}
				\column{.1\textwidth}
				\column{.4\textwidth}
			  	\begin{itemize}
						\item Anwendungsbereiche
							\begin{itemize}
								\item Medizin
								\item Automotive, Logistik
								\item Fotografie
								\item ...
							\end{itemize}
						\item Grundlagendisziplinen
							\begin{itemize}
								\item Mathematik und Statistik
								\item Computer science
								\item ...
							\end{itemize}
					\end{itemize}
				\column{.4\textwidth}
				\column{.1\textwidth}
			\end{columns}
			
		\end{frame}

  \section{Projektinhalte}

    \begin{frame}
      \frametitle{Inhalte}

      \begin{itemize}
        \item Grundlagen der Bildverarbeitung
          \begin{itemize}
            \item Mathematisches Bildmodell
            \item Transformation von Bildern
            \item Merkmale von Bildern und Merkmalsextraktion
            \item ...
          \end{itemize}
        \item Programmierung
          \begin{itemize}
            \item Erste Schritte in Python und OpenCV
            \item Implementierung eines Algorithmus zum image stitching
          \end{itemize}
      \end{itemize}

    \end{frame}

  \section{Ziele des Projekts}

    \begin{frame}
      \frametitle{Ziele des Projekts}
      \begin{columns}
        \begin{column}{0.48\textwidth}
          \begin{figure}
            \includegraphics[scale = 0.53]{images/projektziele.jpg}
          \end{figure}
        \end{column}
        \begin{column}{0.5\textwidth}
          \begin{itemize}
            \item Erlangen von mathematischen und technischen Grundkenntnissen der Bildverarbeitung
            \item Grundlegende Programmierkenntnisse in Python
            \item Python-Programm zum stitching von zwei oder mehreren Bildern
          \end{itemize}
        \end{column}
      \end{columns}
    \end{frame}

\end{document}
