\documentclass[a4paper, headsepline, halfparskip, fleqn, 10pt]{scrartcl}  
    
\usepackage[ngerman]{babel}
\usepackage[latin1]{inputenc}
\usepackage[T1]{fontenc}
\usepackage{ae,aecompl}
\usepackage{xcolor}

\begin{document}
    \section*{Computer vision}
    \subsubsection*{Panorama image stitching}

    Hast du dich schon einmal gefragt, wie es gelingt am Smartphone Panoramabilder zu erstellen? Landschaften, 
		Stadtansichten oder Gruppenfotos - mit passenden Methoden aus dem Bereich Mathematik und Bildverarbeitung lassen sich wie durch 
		Geisterhand mehrere Bilder zu einem Gesamtbild zusammenf{\"u}gen.

		In diesem Projekt besch{\"a}ftigen wir uns mit \textit{panorama image stitching} - dem Verschmelzen von mehreren Einzelaufnahmen
		zu einem gro{\ss}en Ganzen. Wir besprechen mathematische und technische Grundlagen der Bildverarbeitung und implementieren 
		schlie{\ss}lich unsere eigenen Stitching-Algorithmus in Python unter Zuhilfenahme des Programmpakets OpenCV.
		
\end{document}
