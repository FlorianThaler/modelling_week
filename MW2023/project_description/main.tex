\documentclass[
    a4paper,     %% defines the paper size: a4paper (default), a5paper, letterpaper, ...
    % landscape,   %% sets the orientation to landscape
    % twoside,     %% changes to a two-page-layout (alternatively: oneside)
    % twocolumn,   %% changes to a two-column-layout
        headsepline, %% add a horizontal line below the column title
    % footsepline, %% add a horizontal line above the page footer
    % titlepage,   %% only the titlepage (using titlepage-environment) appears on the first page (alternatively: notitlepage)
        halfparskip,     %% insert an empty line between two paragraphs (alternatively: halfparskip, ...)
    % leqno,       %% equation numbers left (instead of right)
        fleqn,       %% equation left-justified (instead of centered)
    % tablecaptionabove, %% captions of tables are above the tables (alternatively: tablecaptionbelow)
    % draft,       %% produce only a draft version (mark lines that need manual edition and don't show graphics)
    % 10pt         %% set default font size to 10 point
    % 11pt         %% set default font size to 11 point
    10pt         %% set default font size to 12 point
    ]{scrartcl}  %% article, see KOMA documentation (scrguide.dvi)
    
    
    
%%%%%%%%%%%%%%%%%%%%%%%%%%%%%%%%%%%%%%%%%%%%%%%%%%%%%%%%%%%%%%%%%%%%%%%%%%%%%%%%
%%%
%%% packages
%%%
        
%%%
%%% encoding and language set
%%%

\usepackage[ngerman]{babel}

%%% inputenc: coding of german special characters
\usepackage[latin1]{inputenc}

%%% fontenc, ae, aecompl: coding of characters in PDF documents
\usepackage[T1]{fontenc}
\usepackage{ae,aecompl}

\usepackage{xcolor}

\begin{document}
    \section*{Mathematische Spieltheorie}
    \subsubsection*{Konkurrenz vs. Kooperation}

    Was haben das Elfmeterschie{\ss}en im Fu{\ss}ball, der Kampf der Geschlechter und Nachbarschaftsstreitigkeiten 
    gemein? Auf den ersten Blick nicht viel wie es scheint. All diese Situationen/Probleme jedoch lassen sich
    als sog. strategische Spiele modellieren und mit Werkzeugen aus dem Bereich der mathematischen Spieltheorie
    analysieren.
    \newline
    Wir werden in den kommenden Tagen auf den Spuren von John Nash und John v. Neumann wandeln und uns mit  
    mathematischer Spieltheorie befassen. Mit den Konzepten die wir in diesem Zusammenhang kennenlernen, werden 
    wir versuchen die oben genannten Situationen zu diskutieren und optimale Strategien f{\"u}r die 
    beteiligten Parteien/Spieler abzuleiten. Ganz besonders interessiert uns dabei die Frage ob die aus subjektiver Sicht optimale
    Strategie eines Spieler tats{\"a}chlich die beste aller m{\"o}glichen Strategien ist. Kann kooperatives Verhalten den Spielausgang
    f{\"u}r beide Spieler positiv beeinflussen?
\end{document}