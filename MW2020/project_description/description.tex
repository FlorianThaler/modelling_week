\documentclass[a4paper,10pt]{article}

% ---------------------------------------------
% --- preamble ---
% ---------------------------------------------


\usepackage[utf8]{inputenc}
\usepackage[english]{babel}

\usepackage{hyperref}
\usepackage{xcolor}

\usepackage{amsmath}
\usepackage{amssymb}
\usepackage{amsthm}

\usepackage{csquotes}

\usepackage{algorithm2e}
\SetAlFnt{\footnotesize}

% Title Page
% \title{}
% \author{}

% ---------------------------------------------
% --- custom theorems ---
% ---------------------------------------------

\theoremstyle{plain}
\newtheorem{myThm}{Theorem}[section]
\newtheorem{myLmm}{Lemma}[section]
\newtheorem{myCrllry}{Corollary}[section]
\newtheorem{myAssmptn}{Assumption}

\theoremstyle{definition}
\newtheorem{myDfn}{Definition}[section]

\theoremstyle{remark}
\newtheorem{myRmrk}{Remark}[section]
\newtheorem{myExmpl}{Example}[section]

% ---------------------------------------------
% --- custom commands ---
% ---------------------------------------------

\newcommand{\nats}{\mathbb{N}}
\newcommand{\reals}{\mathbb{R}}

\DeclareMathOperator*{\argmax}{argmax}
\DeclareMathOperator*{\argmin}{argmin}

\DeclareMathOperator*{\argsup}{argsup}
\DeclareMathOperator*{\arginf}{arginf}


\begin{document}
	Projekt: Kontrolltheorie
	
	Betreuer: Florian Thaler
	
	Autonome Steuerung einer Mondlandefähre
	\newline
	\newline
	\noindent
	\textquote{Houston, Tranquility Base here. The Eagle has landed.} ... Am 20. Juli 1969 landete (anscheinend ;)) die Apollo-Mondlandefähre auf unserem Trabanten. Nach dem Aufsetzen meldete Neil Armstrong die erfolgreiche Landung mit ebendiesen Worten. Ob die Mannschaft um Neil Armstrong tatsächlich auf dem Mond war, werden wir wohl nicht klären können - weshalb wir uns mit dieser Frage auch nur am Rande beschäftigen werden. 
	\newline
	Vielmehr geht es uns darum eine Strategie zur autonomen - also selbstständigen - Steuerung einer Mondlandefähre abzuleiten, welche ein sanftes Aufsetzen dieser auf der Mondoberfläche gewährleistet. Hierfür gilt es zunächst eine physikalische Beschreibung der Problemstellung zu finden und sich mit dieser vertraut zu machen. Die resultierenden Bewegungsgleichungen erlauben es uns dann ein mathematisches Steuerungsproblem zu formulieren, welches der Ausgangspunkt all unserer Überlegungen sein wird.
	\newline
	Es soll uns aber nicht nur darum gehen bereits bekannte Konzepte aus dem Bereich der Kontrolltheorie zu studieren und anzuwenden. Vielmehr bietet die Problemstellung sehr viel Freiraum für eigenständig erarbeitete und kreative Ansätze.
\end{document}